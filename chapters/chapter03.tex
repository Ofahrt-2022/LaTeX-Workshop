\documentclass[
    ngerman,
    accentcolor=3b,
    dark_mode,
    fontsize= 12pt,
    a4paper,
    aspectratio=169,
    colorback=true,
    fancy_row_colors,
    leqno,
    fleqn,
    boxarc=3pt,
    fleqn,
    % shell_escape = false, % Kompatibilität mit sharelatex
]{algoslides}

%%------------%%
%%--Packages--%%
%%------------%%

% \usepackage{audutils}
% \usepackage{fopbot}
\usepackage{booktabs}
\usepackage{tipa}

% Import all Packages from Main Preamble with relative Path (buggy, list packages instead)
% \subimport*{../../}{preamble}

%%--------------------------%%
%%--Imports from Main File--%%
%%--------------------------%%

% Get Labels from Main Document using the xr-hyper Package
\externaldocument[ext:]{../main}
% Set Graphics Path, so pictures load correctly
\graphicspath{{../pictures}}
\def\codeDir{../code}

\begin{document}
    \section{\LaTeX-Basics}\label{2}\label{LaTeX-Basics}
    \subsection{Syntax}
    \begin{frame}[fragile]
        \slidehead{}
        \begin{itemize}
            \item Zeichen mit besonderer Bedeutung: \verb+\+, \verb+&+, \verb+#+, \verb+$+, \verb+%+, \verb+~+, \verb+^+, \verb+_+, \verb+{+, \verb+}+, \verb+[+ und \verb+]+
            \item Kommandos beginnen mit einem Backslask: \verb+\+
            \item Kommandonamen enthalten keine Zahlen, Leerzeichen und Sonderzeichen
            \item Argumente werden in geschweiften Klammern angegeben \verb+{}+
            \item Optionale Argumente werden in eckigen Klammern angegeben \verb+[]+
            \item Environments werden mit \verb+\begin{<name>}+ und \verb+\end{<name>}+ angegeben
            \item Kommentare beginnen mit einem \verb+%+
            \item Parameter werden mit \verb+#+ angegeben und gehen von 1-9\begin{itemize}
                    \item Dabei wird für innere Argumente die Anzahl der \verb+#+ verdoppelt (z.B. \verb+#1+, \verb+##1+, \verb+####1+)
                \end{itemize}
        \end{itemize}
    \end{frame}
    \subsection{Aufbau eines Dokumentes}\label{2.1}\label{2.1}
    \begin{frame}[fragile]
        \slidehead{}
        \begin{columns}[c]
            \begin{column}{.5\textwidth}
                \inputCode[]{minted language=latex,title=\codeBlockTitle{Minimales \LaTeX-Dokument}}{\codeDir/minimal.tex}
            \end{column}%
            \begin{column}{.5\textwidth}
                \centering
                \colorbox{white}{%
                    \color{black}
                    \includegraphics[width=\textwidth,height=5cm,keepaspectratio]{\codeDir/minimal}
                }
            \end{column}
        \end{columns}
    \end{frame}
    \begin{frame}[fragile]
        \slidehead{}
        \begin{columns}[c]
            \begin{column}{.5\textwidth}
                \inputCode[]{minted language=latex,title=\codeBlockTitle{Minimales \LaTeX-Dokument}}{\codeDir/minimal-lipsum.tex}
            \end{column}%
            \begin{column}{.5\textwidth}
                \centering
                \colorbox{white}{%
                    \color{black}
                    \includegraphics[width=\textwidth,height=5cm,keepaspectratio]{\codeDir/minimal-lipsum}
                }
            \end{column}
        \end{columns}
    \end{frame}
    \subsection{Overleaf-Instanz}\label{2.2}\label{2.2}
    \begin{frame}
        \slidehead{}
        \begin{itemize}
            \item Ofahrt-Intern: \url{latex.ofahrt2022.de}
            \item TU-Darmstadt: \url{sharelatex01.ca.hrz.tu-darmstadt.de/}
        \end{itemize}
        
    \end{frame}
    % \subsection{Formeln}
    % \begin{frame}[fragile]
    %     \slidehead{}
    %     \begin{columns}[c]
    %         \begin{column}{.5\textwidth}
                
    %         \end{column}%
    %         \begin{column}{.5\textwidth}
    %             \centering
    %             \colorbox{white}{%
    %                 \color{black}
    %                 \includegraphics[width=\textwidth,height=5cm,keepaspectratio]{\codeDir/minimal-lipsum}
    %             }
    %         \end{column}
    %     \end{columns}
    % \end{frame}
\end{document}