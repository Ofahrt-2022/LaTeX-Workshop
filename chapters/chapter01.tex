\documentclass[
    ngerman,
    accentcolor=3b,
    dark_mode,
    fontsize= 12pt,
    a4paper,
    aspectratio=169,
    colorback=true,
    fancy_row_colors,
    leqno,
    fleqn,
    boxarc=3pt,
    fleqn,
    % shell_escape = false, % Kompatibilität mit sharelatex
]{algoslides}

%%------------%%
%%--Packages--%%
%%------------%%

% \usepackage{audutils}
% \usepackage{fopbot}
\usepackage{booktabs}
\usepackage{tipa}

% Import all Packages from Main Preamble with relative Path (buggy, list packages instead)
% \subimport*{../../}{preamble}

%%--------------------------%%
%%--Imports from Main File--%%
%%--------------------------%%

% Get Labels from Main Document using the xr-hyper Package
\externaldocument[ext:]{../main}
% Set Graphics Path, so pictures load correctly
\graphicspath{{../pictures}}

\begin{document}
    \section{Einführung}\label{1}\label{Einfuehrung}
    \subsection{Was ist \LaTeX?}\label{1.1}\label{1.1}
    \begin{frame}[c]
        \slidehead{}
        \centering\fontsize{40pt}{45pt}\selectfont\LaTeX

        %ˈlaːtɛç
        \medskip\normalsize\textipa{[\textprimstress{}latE\c{c}]}

        \vfill
        LaTeX ist eine Programmiersprache zur Erstellung von PDF-Dokumenten. Erstveröffentlichung 1984
    \end{frame}
    \subsection{Idee von \LaTeX}\label{1.2}\label{1.2}
    \begin{frame}
        \slidehead{}
        \begin{itemize}
            \item Trennung von Inhalt und Design
            \item Häufig verwendeten Code in wiederverwendbaren Commands zusammenfassen
            \item Man erhält \fatsf{exakt} das, was man hinschreibt
            \item Der \LaTeX-Quellcode wird zu PDF (oder SVG) kompiliert
        \end{itemize}
    \end{frame}
    \subsection{Anwendungsmöglichkeiten}
    \begin{frame}
        \slidehead{}
        \begin{itemize}
            \item Arbeiten (z.B. Thesis)
            \item Hausübungsabgaben
            \item Zusammenfassungen
            \item Präsentationen (wie diese)
            \item Handouts
            \item Wissenschaftliches Zeichnen (z.B. Schaltplan, Diagramme, Moleküle, $\dots$)
            \item Vektorgrafiken
            \item Musik komponieren
            \item $\dots$
        \end{itemize}
    \end{frame}
\end{document}
